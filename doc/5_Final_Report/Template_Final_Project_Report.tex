\documentclass{scrarticle}
\usepackage{hyperref}
\usepackage{titling}
\pagenumbering{arabic}

\begin{document}
\title{$\langle$Project Name$\rangle$} %TODO: Fill in project name
\author{%TODO: Fill in your names
	LastName1, FirstName1 (contact email) 
	\and \\
	LastName2, FirstName2 (contact email)
%TODO uncomment for more authors
%	\and \\
%	LastName3, FirstName3
%	\and \\
%	LastName4, FirstName4
}

\begin {center}

\vspace{1 in}

{\Huge Final Project Report}

\vspace {2 in}

{\Huge \textbf{\thetitle}} \\ 
\vspace{1 in}
{\Huge Blockchain (Project) Lab 2022} \\
{\huge by Prof. Dr. Sebastian Faust} \\

\vspace{0.5 in}

{\Large \today}

\vspace{1,5 in}

{\Large \textit \theauthor} 
\end {center}

\clearpage

\paragraph{Abstract.}
\begin{abstract}
	Summarize the motivation and main concepts of your project (1/2 page). This should be an improvement of your previous abstract.
\end{abstract}

\newpage

\textit{The length of this report (from here on) should be at least 7 pages (10 pages for project lab groups) of text (not counting pictures and diagrams). Nevertheless, we would like you to back up your explanations with graphics. The major part of your report is the concept section. This should be at least 4 pages. The implementation section can be quite short, as you will probably explain your implementation in a separate documentation.}

\section{Introduction}
Introduce the reader to your project by explaining your motivation and goals. If there already exist other similar projects, explain the difference between your project and the others.

\section{Concept}
Explain the concepts of your project: System- and software architecture, typical workflows, security considerations, possible efficiency improvements, …. Take a special focus on explaining the projected features of your project and their benefit for your customers.

It is not necessary that all aspects of your concept have been implemented. If you have not realized all modules of your architecture or if you have already planned further improvements without realizing them, feel free to add them to your concept as well.


\section{Implementation}
In this section you should describe which parts of your concept and which features have been implemented and how you have implemented them, e.g. the tools and libraries you have used to realize different modules of your architecture.

We do not want you to provide a documentation of your source code here. If your supervisor wants you to submit a documentation, you should provide it in a separate document.


\section{Roadmap}

Present the timeline of your project (completed work packages and reached milestones) and the difference between the actual timeline and the initially defined roadmap.

\section{Reflection}
Please reflect your project, \textbf{both} in regard to your \textit{organization} and \textbf{project execution}. Try to answer the following questions:
\begin{itemize}
	\item What did go well?
	\item What did not go well?
	\item What would you do different if you could start over?
\end{itemize}

\section{Conclusion}
In this section, you are supposed to assess your goal attainment and discuss advantages and disadvantages (if any) of your result. Additionally, you should provide an overview of open problems and the next steps that should or could be taken by someone continuing with the project.


\newpage
\bibliographystyle{unsrt}
\bibliography{ref}

\end{document}
